\section{Tuning}\label{sec:tuning}
We haven't yet spent much time tuning the code serially. The main thing we did
to improve the serial code was to align memory used in \texttt{lnew} and
\texttt{l} along 64-byte boundaries. This allows us to use \texttt{\#pragma
vector aligned} so code is vectorized using aligned instructions instead of
unaligned instructions which are less efficient.

{
\scriptsize
\begin{verbatim}
LOOP BEGIN at path.c(81,5) inlined into path.c(234,5)
   remark #15389: vectorization support: reference l has unaligned access   [ path.c(83,13) ]
   remark #15389: vectorization support: reference l has unaligned access   [ path.c(83,13) ]
   remark #15381: vectorization support: unaligned access used inside loop body
   remark #15300: LOOP WAS VECTORIZED
   remark #15442: entire loop may be executed in remainder
   remark #15450: unmasked unaligned unit stride loads: 1
   remark #15451: unmasked unaligned unit stride stores: 1
   remark #15475: --- begin vector loop cost summary ---
   remark #15476: scalar loop cost: 14
   remark #15477: vector loop cost: 1.750
   remark #15478: estimated potential speedup: 6.140
   remark #15479: lightweight vector operations: 10
   remark #15488: --- end vector loop cost summary ---
LOOP END
\end{verbatim}
}
