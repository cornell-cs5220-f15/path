\section{Profiling}\label{sec:profiling}
The release implementation of all-pairs shortest paths is a naive
implementation of a repeated squares algorithm that uses OpenMP for
parallelization.  We profiled the release code using VTune Amplifier and the
vectorization reports produced by \icc{}.

\subsection{VTune Amplifier}
The profiling results produced by VTune Amplifier are shown in \figref{vtune-a}
and \figref{vtune-b}. The profiles show that the release implementation spends
most of its time inside the \texttt{square} function. The majority of the time
in \texttt{square} is spent accessing the \texttt{l} array. This is in contrast
to many applications that spend most of their time on computation. This is
unsurprising as the only computation done in the release implementation is a
single addition and comparison.

These results suggested that we optimize our implementations to have good memory
access patterns to take full advantage of the cache as much as possible. This
optimization is discussed in detail in \secref{algo}.

\begin{figure}[h]
\centering
{
\begin{BVerbatim}
Function                  Module       CPU Time
------------------------  -----------  --------
square                    omp.x         41.807s
__kmp_barrier             libiomp5.so   13.993s
__kmpc_reduce_nowait      libiomp5.so    6.397s
__kmp_fork_barrier        libiomp5.so    2.962s
__intel_ssse3_rep_memcpy  omp.x          0.040s
fletcher16                omp.x          0.030s
gen_graph                 omp.x          0.020s
__kmp_join_call           libiomp5.so    0.010s
genrand                   omp.x          0.010s
\end{BVerbatim}
}
\caption{%
  A snippet of the vectorization report produced by VTune Amplifier that shows
  the longest running functions.
}
\label{fig:vtune-a}
\end{figure}

\begin{figure}[h]
{
\scriptsize
\begin{BVerbatim}
Source Line  Source                                                              CPU Time
-----------  ------------------------------------------------------------------  --------
41           int square(int n,               // Number of nodes
42                      int* restrict l,     // Partial distance at step s
43                      int* restrict lnew)  // Partial distance at step s+1
44           {
45               int done = 1;
46               #pragma omp parallel for shared(l, lnew) reduction(&& : done)
47               for (int j = 0; j < n; ++j) {
48                   for (int i = 0; i < n; ++i) {                                 0.020s
49                       int lij = lnew[j*n+i];                                    0.030s
50                       for (int k = 0; k < n; ++k) {                             8.072s
51                           int lik = l[k*n+i];                                  25.646s
52                           int lkj = l[j*n+k];
53                           if (lik + lkj < lij) {                                3.644s
54                               lij = lik+lkj;
55                               done = 0;                                         4.395s
56                           }
57                       }
58                       lnew[j*n+i] = lij;
59                   }
60               }
61               return done;
62           }
\end{BVerbatim}
}
\caption{%
  A snippet of the vectorization report produced by VTune Amplifier that shows
  the longest running sections of the \texttt{square} function.
}
\label{fig:vtune-b}
\end{figure}

\subsection{Vectorization Reports}
A snippet of an \icc{} vectorization report is shown in
\figref{icc-vec-report}. The report shows that most loops in the release
implementation are vectorized. However, the vectorized loops have unaligned
memory accesses because \texttt{l} and \texttt{lnew} are not allocated into
aligned memory. This suggested that we optimized our code to allocate memory
at aligned addresses.

\begin{figure}[h]
\centering
{
\scriptsize
\begin{BVerbatim}
LOOP BEGIN at path.c(108,5) inlined into path.c(258,5)
   remark #15388: vectorization support: reference l has aligned access   [ path.c(110,13) ]
   remark #15388: vectorization support: reference l has aligned access   [ path.c(110,13) ]
   remark #15300: LOOP WAS VECTORIZED
   remark #15448: unmasked aligned unit stride loads: 2
   remark #15449: unmasked aligned unit stride stores: 1
   remark #15475: --- begin vector loop cost summary ---
   remark #15476: scalar loop cost: 14
   remark #15477: vector loop cost: 1.250
   remark #15478: estimated potential speedup: 9.830
   remark #15479: lightweight vector operations: 10
   remark #15488: --- end vector loop cost summary ---
LOOP END
\end{BVerbatim}
}
\caption{\icc{} vectorization report.}
\label{fig:icc-vec-report}
\end{figure}
