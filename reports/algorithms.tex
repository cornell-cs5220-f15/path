\section{Algorithms}\label{sec:algo}
In this section, we describe the three algorithms we have implemented to compute
all-pairs shortest paths.

\subsection{Repeated Squares}
The release implementation uses a simple repeated squares (\rs) dynamic
programming algorithm. Let $l_{ij}^s$ represent the length of the shortest path
from vertex $i$ to vertex $j$ of at most length $2^s$. \rs{} relies on the
following recurrence:
\[
  l_{ij}^{s+1} = \min_k \group{l_{ik}^s + l_{kj}^s}
\]

The base case $l_{ij}^0$ is the weight of the edge from vertex $i$ to vertex
$j$ or $\infty$ if no such edge exists. This recurrence is nearly identical to
the formula used to compute the square of a matrix $A$:
\[
  a_{ij}^2 = \sum_k a_{ik} a_{kj}
\]

The algorithm initializes a distance matrix $L^0$ and iteratively computes
$L^{s+1}$ by squaring $L^s$. The algorithm terminates once squaring $L$ reaches
a fixpoint; that is, once $L^2 = L$. Each squaring requires $O(N^3)$ operations
where $N$ is the number of vertices in the graph and the side-length of $L$. A
shortest path can be of at most length $N$, so the algorithm terminates after
at most $\log N$ iterations. Thus, the worst-case running time of \rs{} is
$O(N^3 \log N)$.

\subsection{Blocked Repeated Squares}
The release implementation of \rs{} uses a naive matrix multiplication kernel
to square $L$. We developed an optimized matrix multiplication kernel that
implements a variety of optimizations. We use this kernel in an optimized
repeated squares implementation that we call \block{}. In this subsection, we
describe the optimizations implemented by \block{}.

\paragraph{Blocking}
A naive matrix multiplication kernel has very poor cache locality. \block{}
implements a blocked matrix multiplication to greatly improve cache locality.
The optimal block size of 128 was found empirically.

\paragraph{Copy Optimization}
When \block{} multiplies two sub-blocks, it first copies them into a smaller,
aligned buffer, which allows the compiler to more aggressively
vectorize loops over the sub-blocks and it makes memory accesses to the
sub-blocks much more regular.

\paragraph{Compile-Time Loop Bounds}
If the size of a matrix is not divisible by the size of a block, then not all
sub-blocks have the same size, and the size of a sub-blocks is only known
at runtime. When \block{} multiplies two sub-blocks, it first checks to see if
they are full-sized blocks, and if they are, it multiplies them using a kernel
whose loop bounds are known at compile-time. By knowing loop-bounds at compile
time, the compiler more aggressively optimizes this code path.  If the
sub-blocks are not full-sized blocks, then it multiplies them using loops with
bounds known at runtime. Since most blocks are full-sized blocks, \block{}
often performs block multiplication with the fully optimized kernel.

\paragraph{Vectorization}
\block{} organizes all loops and array accesses to be fully vectorized.

\subsection{Floyd-Warshall}
The Floyd-Warshall algorithm (\fw{}) is a dynamic programming algorithm
developed by Robert Floyd and Stephen Warshall in the 1960's. Consider a
directed graph  with $N$ nodes ordered $1, \ldots, N$. Let $l_{i,j}^k$ be the
length of the shortest path from vertex $i$ to vertex $j$ using only
intermediate nodes from $\set{1, \ldots, k}$ or $\infty$ if no such path
exists. \fw{} relies on the following recurrence:
\[
  l_{i,j}^{k+1} = \min \group{l_{i,j}^k, l_{i,k+1}^k + l_{k+1,j}^k}
\]

The base case $l_{i,j}^0$ is the weight of the edge from vertex $i$ to vertex
$j$ or $\infty$ if no such edge exists. \fw{} initializes $L^0$ and
iteratively computes $L^{k+1}$ from $L^{k}$ using the above recurrence.
$L^{k+1}$ can be computed in $O(N^2)$ time and the algorithm terminates when it
computes $L^N$. Thus, the algorithm runs in $O(N^3)$ time.
