\section{Profiling}\label{sec:profiling}
\subsection{Identify bottlenecks}
To define the bottlenecks of \ttt{path.C}, we profiled our code
using Intel's VTune Amplifier to understand which part of the code takes the most time to run. VTune Amplifier commands is shown in \figref{amplxe-command}.

\begin{figure}[h]
	\footnotesize
	\begin{verbatim}
	amplxe-cl -collect advanced-hotspots ./path
    amplxe-cl -R hotspots -report-output vtune-report.csv -format csv -csv-delimiter comma
	\end{verbatim}
	\caption{VTune Amplifier Command}
	\label{fig:amplxe-command}
\end{figure}

\subsection{Initial Timing Result}
As shown in \figref{initial-profile-result}, the majority of time is spent inside the \ttt{sqaure} function. We will take about how we optimize the performance of the \ttt{square} function in \secref{vectorization}.  

\begin{figure}[h]
	\footnotesize
	\begin{verbatim}
	Function                     Module        CPU Time		CPU Time:Ideal
	-----------------------------------------  --------     --------------
	square			           	 path.x         45.14s			39.03s
	_kmpbarrier                  libiomp5.so    6.485s			 0.16s
	_kmpc_reduce_nowait          libiomp5.so    2.979s			 	0s
	_kmp_fork_barrier            libiomp5.so    2.553s			 0.03s
	gen_graph             		 path.x         0.030s				0s
	_intel_ssse3_memcpy          path.x         0.030s				0s
	fletcher16                   path.x         0.030s				0s
	\end{verbatim}
	\caption{Initial Profile Result}
	\label{fig:initial-profile-result}
\end{figure}
