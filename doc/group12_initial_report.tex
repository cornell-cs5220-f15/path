\documentclass{article}
\usepackage{graphicx}
\usepackage[colorlinks=true]{hyperref}

\title{CS 5220 Homework 3 Initial Report - Group 12}
\author{Amiraj Dhawan (ad867)
		Weici Hu (wh343)\\
        Sijia Ma (sm2462)}

\begin{document}
\maketitle

\section{Profiling of Serial Performance}
The table below summarizes the profiling result of the serial code. As expected, function square takes up the most of the run-time.
\begin{center}\label{t1}
    \begin{tabular}{ | l | l | l |}
    \hline
    function & CPU time(s)  \\ \hline
    square\_step & 27.519 \\ \hline
    fletcher16 & 0.020 \\ \hline
    \_\_intel\_ssse3\_rep\_memcpy & 0.01\\ \hline
    gen\_graph & 0.01\\ \hline
    genrand & 0.01\\ \hline
    \end{tabular}
\end{center}
\section{Parallel using openMP}
While the code is given by the instructor, we perform strong scaling and plot the result in figure \ref{f1}.
\begin{figure}
	\centering
  \includegraphics[width=5cm]{strong_omp.png}
  \caption{}\label{f1}
\end{figure}
\section{Parallel using MPI}
We specify thread 0 as our root which alone runs all the functions except the shortest\_paths\_mpi and square\_stripe. We parallel the outermost loop in square\_stripe by distributing the work to all the non-root thread according to their thread ID. 
 
We also run strong scaling on the mpi code. The result is summarized in figure \ref{2}
\begin{figure}[h!]
	\centering
  \includegraphics[width=5cm]{strong_mpi.png}
  \caption{}\label{2}
\end{figure}

\end{document}