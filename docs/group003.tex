\documentclass[11pt]{article}
\usepackage[letterpaper,left=1in,right=1in,top=1in,bottom=1in]{geometry}

\usepackage[utf8]{inputenc}
\usepackage{listings}
% \usepackage{xcolor}
\usepackage[dvipsnames]{xcolor}
% credit: http://tex.stackexchange.com/questions/68091/how-do-i-add-syntax-coloring-to-my-c-source-code-in-beamer
% https://en.wikibooks.org/wiki/LaTeX/Source_Code_Listings
\lstset{language=C,
        basicstyle=\ttfamily,
        keywordstyle=\color{blue}\ttfamily,
        stringstyle=\color{red}\ttfamily,
        commentstyle=\color{OliveGreen}\ttfamily,
        morecomment=[l][\color{black}]{\#},
        tabsize=2
}

\title{\textbf{CS 5220 Project 1 Final Report}}
\author{Stephen McDowell (sjm324)}
\date{\textbf{September 2015}}

\usepackage{natbib}
\usepackage[miktex]{gnuplottex}
\usepackage{graphicx}
\usepackage{color}
\usepackage{pdfpages}

\usepackage{enumerate}

\newcommand{\tab}{\hspace*{2em}}
\newcommand{\norm}[1]{\lVert#1\rVert}
\newcommand{\sub}{\textsubscript}
\newcommand{\Depth}{2}
\newcommand{\Height}{2}
\newcommand{\Width}{2}

\usepackage{mathtools}
\DeclarePairedDelimiter\ceil{\lceil}{\rceil}
\DeclarePairedDelimiter\floor{\lfloor}{\rfloor}

\usepackage{relsize}
\usepackage{pdfpages}
\usepackage{hyperref}

\def\wl{\par \vspace{\baselineskip}}

\makeatletter
\renewcommand{\maketitle}{\bgroup\setlength{\parindent}{0pt}
\begin{flushleft}
  {\Large \textsc{\@title}}\newline
  \textsc{\@author}
  \rule{\textwidth}{1pt}
\end{flushleft}\egroup
}
\makeatother

\usepackage{fancyhdr}
\pagestyle{fancy}
\rhead{\textsc{Group 003} -- rmc298, sjm324 -- \textsc{Page} \thepage}
\renewcommand{\headrulewidth}{0pt}
\setlength{\headheight}{0.5in}

\title{CS 5220: Project 3 Initial Report}
\author{Group 003: Robert Chiodi (rmc298), Stephen McDowell (sjm324)}

\begin{document}
\thispagestyle{empty}
\maketitle

\section{Introduction}
For the third assignment, we employ techniques we have learned in previous assignments as well as explore the capabilities of the Message Passing Interface (MPI) in the context of the \href{https://en.wikipedia.org/wiki/Floyd\%E2\%80\%93Warshall_algorithm}{\textcolor{blue}{\underline{Floyd-Warshall}}} algorithm for computing all pairwise shortest paths.  As described in the assignment description, this assignment is very similar to the matrix multiplication algorithm we have studied previously in terms of overall problem setup, and only differs in the actual computations being performed.\\

\noindent We begin by focusing our efforts on using MPI to allow computation across multiple nodes on the \emph{totient} cluster, and then discuss optimizations of the algorithm to speedup the computation.  We end with a discussion of offloading the computation to the Xeon Phi's, and a summary of our tactics.\\

\noindent It is worth mentioning at this point that the assignment asked us to implement a parallel MPI algorithm, and then choose between optimizing the given OpenMP implementation and our own MPI implementation.  We choose to do both for two reasons:

\begin{enumerate}[1.]
  \item The optimizations for the OpenMP implementation are very much the same as the MPI implementation in that the algorithm being executed is the same -- the only difference being how threads are delegated work, and
  \item Recent trends in high performance computing fall more along the lines of ``first, use MPI to coordinate between multiple nodes, then use OpenMP on each node to further parallelize the work.''
\end{enumerate}

\section{Optimizations}
There are two general forms of optimizations that we employ:

\begin{enumerate}[1.]
  \item Vectorization and/or compilation optimizations, and
  \item Algorithmic optimizations and/or enhancements.
\end{enumerate}

\noindent Each provide their own flavor of performance increase, but (1) is not as applicable for this assignment as it has been in the past since we do not have a large concern of optimizing the throughput of floating point operations.  As such, after a few modifications, the majority of optimization efforts should be focused on (2).

\subsection{Vectorization and Compilation Strategies}

Though at a first glance there do not seem to be many opportunities for vectorization, we elect to give \texttt{icc} the hints it needs to optimize as much as possible.  To enable portability, the following compilation hints are defined:

\begin{lstlisting}
#ifdef __INTEL_COMPILER
    #define DEF_ALIGN(x) __declspec(align((x)))
    #define USE_ALIGN(var, align) __assume_aligned((var), (align));
#else // GCC
    #define DEF_ALIGN(x) __attribute__ ((aligned((x))))
    /* __builtin_assume_align is unreliabale... */
    #define USE_ALIGN(var, align) ((void)0)
#endif
\end{lstlisting}

\noindent These two must be used in lock-step.  When declaring a variable of a specific alignment, we use \texttt{DEF\_ALIGN} to declare that this variable has a specific alignment.  An example usage would be 
\begin{lstlisting}
  DEF_ALIGN(BYTE_ALIGN)
  int * restrict lnew = (int *)_mm_malloc(num_bytes, BYTE_ALIGN);
\end{lstlisting}

\noindent noting that declaring alignment is not enough, it must be enforced through either aligned allocators or \texttt{\_mm\_malloc}.  However, declaring alignment alone is not enough to enable the compiler to make the proper decisions when arranging loops and vectorizing. For example, when you pass a variable to a function, you need to indicate to the compiler that it has a given alignment, as in

\begin{lstlisting}
int square(int n,                 // Number of nodes
           int * restrict l,      // Partial distance at step s
           int * restrict lnew) { // Partial distance at step s+1

    USE_ALIGN(l,    BYTE_ALIGN);
    USE_ALIGN(lnew, BYTE_ALIGN);
    .
    .
    .
}
\end{lstlisting}

\noindent When used correctly, the \texttt{DEF} -- \texttt{USE} strategy can be quite effective at optimizing, however for this program it does not appear to be too effective.  Comparing to the base code, including this simple two-phase process gives the following increases:
\begin{table}[h]
	\begin{center}
		\begin{tabular}{|c| c| c| c|}
			\hline
			\# of Threads & Basic Time (s) & Optimized Time (s) & \% Increase \\ \hline
			1 &  538.585 & 538.384 & 0.04\%	\\ \hline
			2 &  256.654 &  261.027& -1.70\%	\\ \hline
			4 & 123.608  & 119.715 & 3.15\%	\\ \hline
			8 & 62.932  & 62.926 & 	0.01\%	\\ \hline
			16 & 47.883  & 44.550& 	6.96\%	\\ \hline
			24 & 41.951 & 44.717 & 	-6.593\%	\\ \hline
						
		\end{tabular}
	\end{center}
	\caption{Comparison between base code and optimized code speeds for $n = 4000$ and probability of $p = 0.05$.}
\end{table}


\noindent The only other optimization at this level that we tried was with respect to the branching in the square function:

\begin{lstlisting}
    if (lik + lkj < lij) {
        lij = lik+lkj;
        done = 0;
    }
\end{lstlisting}

\noindent Basic parallel programming techniques dictate that branches are the death of all parallel algorithms -- they induce thread divergence and prevent effective chaining of the parallel algorithm.  Our remedy was to remove the branching using the following strategy:

\begin{lstlisting}
    // https://graphics.stanford.edu/
    //~seander/bithacks.html#IntegerMinOrMax
    // If you know that INT_MIN <= x - y <= INT_MAX,
    // then you can use the following, which are faster
    // because (x - y) only needs to be evaluated once.
    int sum  = lik + lkj;
    int prev = lij;
    // min(sum, lij)
    lij = lij + ((sum - lij) & ((sum - lij) >>
     (sizeof(int) * CHAR_BIT - 1)));
    done = done && sum >= prev;
\end{lstlisting}

\noindent Though this removes the branching in the code, it actually performs significantly worse.  The reason becomes clear after noticing two things.  First, the introduction of significantly more instructions in this loop (as examined by using \texttt{objdump}):

\begin{lstlisting}
216d: 04 7f                 add    $0x7f,%al
216f: 00 0d 0e 73 75 6d     add    %cl,0x6d75730e
2175: 2e 31 32              xor    %esi,%cs:(%rdx
2178: 34 32                 xor    $0x32,%al
217a: 5f                    pop    %rdi
217b: 56                    push   %rsi
217c: 24 31                 and    $0x31,%al
217e: 65 00 0a              add    %cl,%gs:(%rdx)
2181: 04 00                 add    $0x0,%al
2183: 04 00                 add    $0x0,%al
2185: 00 56 6e              add    %dl,0x6e(%rsi)
2188: 82                    (bad)
2189: 88 00                 mov    %al,(%rax)
218b: 00 06                 add    %al,(%rsi)
218d: 04 7f                 add    $0x7f,%al
218f: 00 0e                 add    %cl,(%rsi)
2191: 0f 70 72 65 76        pshufw $0x76,0x65(%rd
2196: 2e 31 32              xor    %esi,%cs:(%rdx
2199: 34 32                 xor    $0x32,%al
219b: 5f                    pop    %rdi
219c: 56                    push   %rsi
219d: 24 31                 and    $0x31,%al
219f: 66                    data16
21a0: 00 0b                 add    %cl,(%rbx)
21a2: 04 00                 add    $0x0,%al
21a4: 04 00                 add    $0x0,%al
21a6: 00 56 6e              add    %dl,0x6e(%rsi)
21a9: 82                    (bad)
21aa: 88 00                 mov    %al,(%rax)
21ac: 00 06                 add    %al,(%rsi)
21ae: 04 7f                 add    $0x7f,%al
21b0: 01 4d 5f              add    %ecx,0x5f(%rbp
21b3: 00 01                 add    %al,(%rcx)
21b5: 00 00                 add    %al,(%rax)
21b7: 00 02                 add    %al,(%rdx)
21b9: 4d 5f                 rex.WRB pop %r15
21bb: 01 02                 add    %eax,(%rdx)
\end{lstlisting}

\noindent The other element that must be observed here is that x86 and later have instructions for conditional move / set, for which the branch above actually reduces to a small set of simple instructions.  That is, although in code we write a branch using an if statement, it gets reduced to a consistent number of instructions, since the logic inside of the branch is simple enough for the compiler to reduce it to a branchless set of instructions anyway.

\subsection{OpenMP Scaling Study}
In order to analyze the efficiency of parallelizing this shortest paths algorithm with OpenMP, a strong and weak scaling study was performed. This will be used alongside a scaling study for our MPI implementation, presented later, to determine the best method for parallelization. \\

\noindent First, we performed a strong scaling study to see the potential speedup that can be gained by using more processors. For this, we define the strong scaling as
\begin{equation}
\mathrm{Strong \; Scaling} = \frac{t_{\mathrm{serial}}}{t_{\mathrm{parallel}}} \; ,
\label{strongscale}
\end{equation}
and a strong scaling efficiency as 
\begin{equation}
\mathrm{Strong \; Scaling \; Efficiency} = \frac{t_{\mathrm{serial}}}{t_{\mathrm{parallel}}} \frac{1}{p} 
\label{normss}
\end{equation}
where $p$ is the number if processors. To enable easy comparison, we will use the strong scaling efficiency, which is normalized by a linear (ideal) speedup. The speedup of OpenMP derived from our study can be seen in Figure~\ref{omp_ss}.
\begin{figure}[h!]
	\begin{center}
		\begin{gnuplot}[terminal=cairolatex, terminaloptions= color] 
			#set key at 15.8,41
			set size 1.0,0.75              
			unset log                          
			unset label                          
			#set xtic auto offset 0,-0.5 font ",10"                     
			#set ytic 25 font ",10" 
			#Set Info
			set xlabel "Number of Threads"
			#set xlabel offset 0,-0.5
			set ylabel "Scaling Efficiency"
			#set ylabel offset -1.25,0
			#set xr [0.0:16.0]
			set yr [0.0:1.5]
			plot "./benchmarking/vectorized/strong/p025strong.txt" u 1:(489.7618279/$2/$1) t 'p = 0.025' w linespoints, \
			"./benchmarking/vectorized/strong/p05strong.txt" u 1:(538.3848/$2/$1) t 'p = 0.050' w lp, \
			"./benchmarking/vectorized/strong/p10strong.txt" u 1:(359.1135909/$2/$1) t 'p = 0.100' w lp, \
			"./benchmarking/vectorized/strong/p30strong.txt" u 1:(326.293136119/$2/$1) t 'p = 0.300' w lp
		\end{gnuplot}
		\caption{Strong scaling for optimized OpenMP parallelized path algorithm with $\mathtt{n} = 4000$.}
		\label{omp_ss}
	\end{center}
\end{figure}
\noindent While it is strange that the strong scaling efficiency rises over one for several points, this is possibly due to heterogeneity in the cluster. In order to quantify the deviation in time with constant parameters, the program should be run several times, however we have not had the time available to conduct this study. The OpenMP parallelization does seem to show constant scaling efficiency decrease, with no discontinuity between 8 and 16 processors, where threads from two different Xeon chips begin to be used.

\noindent A weak scaling was also performed in order to see the significance of communication for the OpenMP parallelized program. The parameters for this study can be seen in Figure~\ref{mpi_ws_tab}, with each run of the program being done four times for probabilities of 0.025, 0.05, 0.10, and 0.30. The results of this study can be seen in Figure~\ref{omp_ws} where weak scaling is calculated as
\begin{equation}
\mathrm{Weak \; Scaling} = \frac{t_{\mathrm{serial}}(n(p))}{t_{\mathrm{parallel}}(n(p),p)}
\label{ws}
\end{equation}
\begin{table}[h]
	\begin{center}
		\begin{tabular}{|c| c|}
			\hline
			\# of Threads & n \\ \hline
			1 & 1000 \\ \hline
			2 & 1414 \\ \hline
			4 & 2000 \\ \hline
			8 & 2828 \\ \hline
			16 & 4000 \\ \hline
			24 & 4899 \\ \hline
		\end{tabular}
	\end{center}
	\caption{Table of weak scaling parameters for number of threads and the squart root of the problem size ($n$)}
	\label{mpi_ws_tab}
\end{table}

\begin{figure}[h!]
	\begin{center}
		\begin{gnuplot}[terminal=cairolatex, terminaloptions= color] 
			#set key at 15.8,41
			set size 1.0,0.75              
			unset log                          
			unset label                          
			#set xtic auto offset 0,-0.5 font ",10"                     
			#set ytic 25 font ",10" 
			#Set Info
			set xlabel "Number of Threads"
			#set xlabel offset 0,-0.5
			set ylabel "Weak Scaling"
			#set ylabel offset -1.25,0
			#set xr [0.0:16.0]
			set yr [0.0:1.4]
			plot "./benchmarking/vectorized/weak/p025weak.txt" u 1:(2.9435491/$2) t 'p = 0.025' w linespoints, \
			"./benchmarking/vectorized/weak/p05weak.txt" u 1:(2.44597/$2) t 'p = 0.050' w lp, \
			"./benchmarking/vectorized/weak/p10weak.txt" u 1:(3.01424789428/$2) t 'p = 0.100' w lp, \
			"./benchmarking/vectorized/weak/p30weak.txt" u 1:(1.6850919/$2) t 'p = 0.300' w lp
		\end{gnuplot}
		\caption{Weak scaling for optimized OpenMP parallelized path algorithm with program parameters given in Table~\ref{mpi_ws_tab}.}
		\label{omp_ws}
	\end{center}
\end{figure}
From Figure~\ref{omp_ws}, it appears that communication and parallel overhead is very costly in the OpenMP shortest paths algorithm. This is most likely due to the fact that the OpenMP team is launched each time inside the \texttt{square} function, which is called multiple times during each program execution. Once again, there is a point displaying super-linear weak scaling. This is due to the fact that there is 302 less nodes per processor, requiring less work per processor. Currently, we are unsure why this only occurs for one of the probabilities. 

\subsection{Algorithmic Optimizations}

\noindent The most immediate and obvious form of algorithmic optimization for this assignment is to employ the \texttt{matmul-} style sub-blocking we used in the first assignment.  This section will be detailed at a later time after we have incorporated this in our program.  Since this technique is familiar at this point, we elected to focus our efforts on MPI and elsewhere for the initial report.\\

\noindent At this time we feel it only worth mentioning that arbitrarily sized sub-blocks are a desirable feature, and that when considering the dimensions of the sub-blocks to employ a simple enhancement can be to have rectangular blocks that are larger in the horizontal than in the vertical.  This enables unit-stride access for a greater number of elements when doing the actual computation.

\section{Message Passing Interface (MPI)}

\noindent The method of MPI parallelization was chosen for its ease of implementation and decent performance. To start, the Floyd-Warshall pathing algorithm, originally parallelized using OpenMP, was changed in order to make use of MPI. First, the parallelization and delegation of work was moved outside of the square function and into the main function, in order to eliminate multiple parallel initialization costs. From this point, the pathing matrix is split into $\mathtt{npx}\times \mathtt{npy}$ subdomains, where \texttt{npx} and \texttt{npy} are the number of processors used to divide the matrix in the horizontal and vertical directions, respectively. \texttt{MPI\_CART} commands are used to organize these threads in an optimal orientation and provide coordinates for the responsibilities of each processor, which are then used to determine the starting and ending indices of each thread's subdomain in relation to the global pathing matrix. At this point, each thread has its own copy of the entire pathing matrix and calls the \texttt{shortest\_path} function. \\

\noindent The functions \texttt{infinitize} and \texttt{deinfinitize}, as well as the diagonal zeroing loop, were not parallelized due to their relatively little computational expense. In doing so, the associated parallel communication costs were avoided. The values of \texttt{l} are then copied into \texttt{lnew} using \texttt{MPI\_ALLREDUCE} with the \texttt{MPI\_MIN} operator. In this way, consistency between processors is ensured. It would also be possible to parallelize \texttt{infinitize} and \texttt{deinfinitize} with this method of communication, if the problem size became large enough to warrant the additional communication costs associated with it. \\

\noindent In the main loop of \texttt{shortest\_path}, the \texttt{square} function is called by each thread, where each thread determines the shortest paths for its subdomain. Once each thread is finished inside square, \texttt{MPI\_ALLREDUCE} is once again used with the \texttt{MPI\_MIN} operator to determine if all processors returned a value of 1, indicating all shortest paths have been found. The array \texttt{lnew} is then communicated and stored in \texttt{l}, once again using \texttt{MPI\_ALLREDUCE} with the \texttt{MPI\_MIN} operator. \\

\noindent While this is not the most efficient MPI implementation, we believe it to be sufficient to show that shared memory OpenMP will be significantly faster than MPI for problem sizes that fit on one node. This is due to the nature of the pathing algorithm, where every point to the left, right, top, and bottom of the element of interest needs to be checked to determine if it provides a shorter path, making this more efficient with shared memory. If the problem sizes driven by an application scaled greatly so that they could no longer fit on one node, the MPI implementation could be rewritten to provide some marginal gains in performance and noticeable gains in memory efficiency, enabling the algorithm to be run on very large problem sizes. \\

\noindent Due to the ability of MPI to utilize several nodes, we believe a heterogeneous parallelization strategy would be ideal, where MPI is used for inter-node communications and OpenMP is otherwise used to compute on the Xeon Phi cores. At this time, however, we have been unable to run on multiple nodes of Totient using MPI, preventing us from implementing this.

\subsection{MPI Scaling Study}
In order to understand the performance of the MPI-parallelized shortest paths algorithm, strong and weak scaling studies were run. For the strong scaling, six simulations were run for a problem size of $4000 \times 4000$ for four separate probabilities: 0.025, 0.05, 0.1, 0.3. The results are plotted in Figure~\ref{mpi_ss}.

\begin{figure}[h]
	\begin{center}
		\begin{gnuplot}[terminal=cairolatex, terminaloptions= color] 
			#set key at 15.8,41
			set size 1.0,0.75              
			unset log                          
			unset label                          
			#set xtic auto offset 0,-0.5 font ",10"                     
			#set ytic 25 font ",10" 
			#Set Info
			set xlabel "Number of Threads"
			#set xlabel offset 0,-0.5
			set ylabel "Scaling Efficiency"
			#set ylabel offset -1.25,0
			#set xr [0.0:16.0]
			set yr [0.0:1.0]
			plot "./benchmarking/mpi/strong/p025strong.txt" u 1:(490.5691/$2/$1) t 'p = 0.025' w linespoints, \
			"./benchmarking/mpi/strong/p05strong.txt" u 1:(538.9763960838318/$2/$1) t 'p = 0.050' w lp, \
			"./benchmarking/mpi/strong/p10strong.txt" u 1:(325.8776059150/$2/$1) t 'p = 0.100' w lp, \
			"./benchmarking/mpi/strong/p30strong.txt" u 1:(325.54634/$2/$1) t 'p = 0.300' w lp
		\end{gnuplot}
		\caption{Strong scaling for MPI parallelized path algorithm with $\mathtt{n} = 4000$.}
		\label{mpi_ss}
	\end{center}
\end{figure}
\noindent A couple of comments can be made about the parallel efficiency shown in Figure~\ref{mpi_ss}. First, there is a stark decrease in scaling efficiency between 8 and 16 threads. This is most likely due to an increase in communication costs, as running the program with 8 threads only requires communication on the same Xeon chip, while 16 threads spans two Xeon chips. For all probabilities presented here, this actually leads to the program completing quicker when run with only 8 processors as opposed to 16. It is interesting to note that this discontinuity in scaling between 8 and 16 processors does not appear for OpenMP (see Figure~\ref{omp_ss}), possibly due to optimized communications inside the OpenMP library itself. A second comment that can be made is that scaling efficiency appears to be independent of the probability parameter. \\

\noindent Typically, weak scaling is used to view the communication costs associated with parallelizing a program since the algorithmic work should remain constant per processor during the study. This should allow us to see if communications become a dominant source of time in our program, requiring us to rethink our MPI parallelization strategy. The parameters for our weak scaling study can be seen in Table~\ref{mpi_ws_tab}, where each parameter configuration was one again run for probabilities of 0.025, 0.05, 0.1, and 0.3. The results of the study are shown in Figure~\ref{mpi_ws}

\begin{figure}[h]
	\begin{center}
		\begin{gnuplot}[terminal=cairolatex, terminaloptions= color] 
			#set key at 15.8,41
			set size 1.0,0.75              
			unset log                          
			unset label                          
			#set xtic auto offset 0,-0.5 font ",10"                     
			#set ytic 25 font ",10" 
			#Set Info
			set xlabel "Number of Threads"
			#set xlabel offset 0,-0.5
			set ylabel "Weak Scaling"
			#set ylabel offset -1.25,0
			#set xr [0.0:16.0]
			set yr [0.0:1.2]
			plot "./benchmarking/mpi/weak/p025weak.txt" u 1:(2.49040075/$2) t 'p = 0.025' w linespoints, \
			"./benchmarking/mpi/weak/p05weak.txt" u 1:(2.4700660705/$2) t 'p = 0.050' w lp, \
			"./benchmarking/mpi/weak/p10weak.txt" u 1:(2.49511981/$2) t 'p = 0.100' w lp, \
			"./benchmarking/mpi/weak/p30weak.txt" u 1:(1.6716897/$2) t 'p = 0.300' w lp
		\end{gnuplot}
		\caption{Weak scaling for MPI parallelized path algorithm with program parameters given in Table~\ref{mpi_ws_tab}.}
		\label{mpi_ws}
	\end{center}
\end{figure}
\noindent From the weak scaling, it can once again be seen that large communication costs are associated with using threads that span two Xeon chips by comparing the weak scaling at 8 threads and 16 threads. The poor performance indicates that if we desire to use the MPI parallelization as anything more than a message passer for OpenMP thread teams running on Xeon Phi's, significant effort should be spent on improving parallelization through reducing the communication frequency and the size of data communicated. The higher than one weak scaling for two threads with a probability of 0.1 is once again due to the fact that there is 302 less nodes in the system per processor. We are still unsure of the cause and do not yet have an explanation for why the super-linear scaling displayed in the MPI version is for a probability of 0.10, while for the OpenMP version it is super-linear for a probability of 0.025. 


\section{Offloading to the Phi's}

The following compilation hint has been defined:

\begin{lstlisting}
#ifdef __MIC__
    #define BYTE_ALIGN 64
#else
    #define BYTE_ALIGN 32
#endif
\end{lstlisting}

\noindent This conditional definition is to account for the different preferred byte alignment of the Node (Xeon) and the Device (Phi).  We need to more carefully inspect the legitimacy of this tactic, because what this will do currently is the following:

\begin{enumerate}[1.]
  \item When the code is compiled for the Node, the byte alignment will be 32.
  \item When the code is compiled for the Device, the byte alignment will be 64.
\end{enumerate}

\noindent This is because, where \texttt{\#pragma offload target(mic)} sections are concerned, the code is compiled in two passes by \texttt{icc} -- once for the Node and once for the Device.  Although the above specified alignments are the respective preferred alignments, we are breaking the rule of enforcement here and will have to revisit how to appropriately do this.  Since we aim to send the majority of the computation on the Phi's, we may elect to define the byte alignment to be 64 to guarantee that the Phi gets its preferences.\\

\noindent Regardless, offloading this algorithm to the Phi's will be straightforward.  As with \texttt{matmul-}, we don't actually need to know anything about our neighboring blocks to carry forward with the computation.  The key will be delegating work evenly, and ensuring that memory is re-used on the Phi as much as possible.  There will invariably be an overhead associated with the synchronization between the two Phi's at each step (e.g. while checking the \texttt{done} flag), however we feel that for sufficiently large problem sizes we will see a good increase in speed by wielding the Phi's to our advantage.\\

\noindent We will employ these tactics after we have finished bringing in the \texttt{matmul-} style sub-blocking into our code.

\section{Summary and Future Work}
Thus far, we have optimized the preliminary code after hand implementing timing into the program for profiling. Using these optimizations while keeping the OpenMP parallelization, we then performed weak and strong scaling studies to assess the parallelization of the algorithm and to have a comparison for an MPI implementation. We then used the optimizations previously incorporated into the OpenMP parallelized framework to parallelize the algorithm with MPI, which forced us to switch to a distributed memory paradigm. In both the strong and weak scaling studies, we have seen that the original, OpenMP parallelized, algorithm exceeds the MPI version. For this reason, we will continue working on improving the OpenMP parallelized algorithm by offloading work to the Xeon Phi's. Once we have this working, we plan to use MPI to work as master processes that are solely responsible for communicating between nodes and offloading work to multiple Xeon Phi's. 


\end{document}
